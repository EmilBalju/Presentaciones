\documentclass[12pt,spanish]{homework}
\usepackage[utf8]{inputenc}
\usepackage[T1]{fontenc}
%\usepackage{mathpazo}
\usepackage{graphicx}
\usepackage{booktabs}
\usepackage{listings}
\usepackage{enumerate} 
\usepackage{amsmath}
\usepackage{amssymb}
\usepackage[spanish,mexico]{babel}
\inputencoding{latin1}

\title{Variables Uniformes Discretas}
\author{Emilio Balan,Amilcar Campos, Elian Carrasco, Citlali Guti�rrez, H�ctor Rodr�guez}
\date{4 de noviembre del 2020}
\institute{Universidad Aut�noma de Yucat�n \\ Facultad de Matem�ticas - UADY}
\class{Probabilidad 1 (Grupo B)}
\professor{Lic. Ernesto Guerrero Lara}

\begin{document}

\maketitle

\section*{Experimento que origina a la variable uniforme discreta}
Es la distribuci�n de probabilidad se asocia a variables cuyos posibles valores tienen todos la misma probabilidad. Si una variable aleatoria X cuyos posibles valores son $({x_{1}, . . . ,x_{n}})$  
tiene distribuci�n uniforme discreta entonces
$$
P(X=x_{1})=P(X=x_{2})=...=P(X=x_{n})={\dfrac{1}{n}}
$$
Decimos que una variable aleatoria X tiene una distribuci�n uniforme discreta sobre el conjunto de n n�meros $\lbrace x_{1}, . . . , x_{n} \rbrace$ si la probabilidad de que
X tome cualquiera de estos valores es constante $\dfrac{1}{n}$. Esta distribuci�n surge
en espacios de probabilidad equiprobables, esto es, en situaciones en donde
tenemos n resultados diferentes y todos ellos tienen la misma probabilidad
de ocurrir.
Los juegos de loter�a son un ejemplo donde puede aplicarse esta distribuci�on de probabilidad.
Se denota c�mo $X \sim$ unif$\lbrace x_{1}, . . ., x_{n} \rbrace$, en donde el s�mbolo $\sim$ se lee "se distribuye como" o "tiene una distribuci�n". 
\section*{Funci�n masa de probabilidad de la variable}
La funci�n de probabilidad de esta variable aleatoria es:
$$
f_{X} (x)=\begin{cases}
{\dfrac{1}{n}} & si \mbox{ $x=x_{1},...,x_{n}$}\\{0 }& \mbox{otro caso}\
\end{cases}
$$
\section*{Calcular la esperanza matem�tica}
La esperanza de esta distribuci�n puede ser obtenida como una media aritm�tica de los valores que toma la variable $\lbrace  x_{1},x_{2},...,x_{n} \rbrace$.
$$E(x)=\sum_{i=1}^{n}x_{i}f_X(x_{i})=\dfrac{1}{n}\sum_{i=1}^{n}x_{i}=\mu$$
\section*{Calcular la varianza}
La varianza se obtiene de la forma ya conocida; es decir, como la varianza de esos mismos valores.Expresada en t�rminos de momentos, la varianza ser�:
$$Var(x)=\dfrac{1}{n}\sum_{i=1}^{n}(x_{i}-\mu)^{2}= \sigma ^{2}$$
\section*{Ejercicios propuestos}

\end{document}